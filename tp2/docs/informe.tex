\documentclass[10pt,a4paper]{article}
\usepackage[paper=a4paper, hmargin=3cm, bottom=2cm, top=2cm]{geometry}

\usepackage[utf8x]{inputenc}
\usepackage[spanish]{babel}

\usepackage{mathtools}
\usepackage{amsmath}
\usepackage{amsfonts}
\usepackage{amssymb}

\usepackage{xcolor}
\usepackage{listingsutf8}
\usepackage{booktabs}
\usepackage{hyperref}
\usepackage{multirow}

\usepackage{caption}
\usepackage{subcaption}

\usepackage{algorithm}
\usepackage[noend]{algpseudocode}

\usepackage{graphicx}
\usepackage{tikz}
\usepackage{relsize}
\usepackage{epstopdf}

\usepackage{bytefield}

\usepackage{tcolorbox}

\DeclarePairedDelimiter{\ceil}{\lceil}{\rceil}

% set the default code style
\lstset{
    frame=tb, % draw a frame at the top and bottom of the code block
    tabsize=4, % tab space width
    showstringspaces=false, % don't mark spaces in strings
    numbers=left, % display line numbers on the left
    commentstyle=\color{green}, % comment color
    keywordstyle=\color{blue}, % keyword color
    stringstyle=\color{red} % string color
}

% mathy stuff
\newtheorem{theorem}{Theorem}[section]
\newtheorem{lemma}[theorem]{Lemma}
\newtheorem{proposition}[theorem]{Proposición}
\newtheorem{corollary}[theorem]{Corollary}

\newenvironment{proof}[1][Demostración]{\begin{trivlist}
\item[\hskip \labelsep {\bfseries #1}]}{\end{trivlist}}
\newenvironment{definition}[1][Definición]{\begin{trivlist}
\item[\hskip \labelsep {\bfseries #1}]}{\end{trivlist}}
\newenvironment{example}[1][Example]{\begin{trivlist}
\item[\hskip \labelsep {\bfseries #1}]}{\end{trivlist}}
\newenvironment{remark}[1][Remark]{\begin{trivlist}
\item[\hskip \labelsep {\bfseries #1}]}{\end{trivlist}}

\newcommand{\qed}{\nobreak \ifvmode \relax \else
      \ifdim\lastskip<1.5em \hskip-\lastskip
      \hskip1.5em plus0em minus0.5em \fi \nobreak
      \vrule height0.75em width0.5em depth0.25em\fi}

\title{Teoría de las Comunicaciones \\ TP2}

\newcommand{\order}[1]{$\mathcal{O}(#1)$}

\begin{document}

%% cover page

\maketitle

\bigskip

\begin{table}[h]
\centering
\begin{tabular}{|l l l|}
\hline
Integrante       & \multicolumn{1}{c}{LU}     & Correo electrónico        \\ \hline
Martín Baigorria & \multicolumn{1}{c}{575/14} & martinbaigorria@gmail.com \\ 
Federico Beuter & 827/13                      & federicobeuter@gmail.com \\
Mauro Cherubini & 835/13                      & cheru.mf@gmail.com \\ \hline
\end{tabular}
\end{table}

\vfill

\begin{center}
\textbf{Reservado para la cátedra}
\end{center}
\begin{table}[h]
\centering
\begin{tabular}{|l|l|l|}
\hline
Instancia       & Docente & Nota \\ \hline
Primera entrega &         &      \\ \hline
Segunda entrega &         &      \\ \hline
\end{tabular}
\end{table}

\newpage
\tableofcontents
\newpage

% end cover page

\section{Introducción}

Cuando un usuario accede a un sitio web o intenta transmitir información de un punto a otro por medio de la web, el mismo en general se abstrae de todos los mecanismos necesarios requeridos para esta transmisión. Dada la gran cantidad y heterogeneidad del hardware subyacente, en lo comienzos de la web se empezaron a diseñar distintos tipos de protocolos de comunicación para lograr unificar todos estos dispositivos en una red capaz de transmitir información desde cualquier punto a otro de forma relativamente confiable.

Hoy en día, el protocolo dominante para la transmisión de datos por la red es el IP (Internet Protocol). Creado por Vint Cerf y Bob Kahn en 1974, el protocolo IP esta diseñado para ser utilizado en redes de conmutación de paquetes y no esta orientado a conexión. Esto significa que al enviar un archivo, el mismo es fragmentado en paquetes que no necesariamente siguen la misma ruta en la red. Además, al no ser un protocolo orientado a conexión, no hay garantías de que los paquetes lleguen a destino dado que no hay ningún tipo de protocolo de handshake entre origen y destino.

En relación al modelo OSI, el protocolo IP pertenece a la capa de internet. Existen muchos otros protocolos que han sido construidos sobre el protocolo IP con el objetivo de proveer otro tipo de garantías y funcionalidades, entre los mas conocidos se encuentran el protocolo TCP (Transmission Control Protocol) y el UDP (User Datagram Protocol). Estos protocolos pertenecen a la capa de interconexión.

En el presente trabajo utilizaremos el protocolo ICMP (Internet Control Message Protocol), el cual esta especificado en el RFC 792 \cite{RFC}, con el objetivo de analizar las diferentes trazas y el RTT (round trip time) al momento de conectarse a un sitio web. Una traza se define como la sucesión de dispositivos de red que son recorridos, ya sean puentes, routers o gateways, al momento de transmitir información en la red. El protocolo ICMP esta implementado sobre IP, aunque se considera que el mismo no pertenece a la capa de interconexión a diferencia de TCP y UDP. Esto se debe a que su principal propósito no es intercambiar datos entre sistemas, si no que en general se utilizan para enviar mensajes de error entre dispositivos de red, con la excepción de su uso en herramientas como ping y traceroute.

\subsection{Traceroute}

La herramienta traceroute, desarrollada inicialmente por Van Jacobson en 1988, es una herramienta sumamente útil de diagnostico de red para buscar una aproximación de la traza de conexión y encontrar el RTT a cada hop (o salto) en la traza. Como ya hemos mencionado, esta herramienta utiliza el protocolo ICMP. A continuación discutiremos la estructura de los paquetes IP/ICMP para luego explicar como se hace efectivamente para identificar los diferentes hops. Luego discutiremos que potenciales problemas puede tener esta herramienta.

Como ya hemos mencionado, el protocolo ICMP esta implementado sobre IP. A continuación mostramos la estructura del header de un paquete IPv4. Todo lo que mencionaremos sigue siendo valido para IPv6.

\begin{figure}[H]
  \vspace{2em}
  \begin{center}
    \begin{bytefield}[bitwidth=1.1em]{32}
      \bitheader{0-31} \\
      \bitbox{4}{Version} & \bitbox{4}{IHL}  & \bitbox{6}{DSCP} & \bitbox{2}{ECN} & \bitbox{16}{Total Length}  \\
      \bitbox{16}{Identification} & \bitbox{3}{Flags}  & \bitbox{13}{Fragment Offset}  \\
      \bitbox{8}{Time To Live} & \bitbox{8}{Protocol}  & \bitbox{16}{Header Checksum}  \\
      \bitbox{32}{\emph{Source IP Address}} \\
      \bitbox{32}{\emph{Destination IP Address}} \\
      \bitbox{32}{\emph{Options (if IHL $>$ 5)}} \\
    \end{bytefield}
  \end{center}
  \caption{Paquete IPv4}
  \label{fig:ipv4-packet}
\end{figure}

Como podemos observador, el header de un paquete IPv4 tiene 24 bytes. En este momento, el campo que mas nos interesa es Time To Live (TTL). Como lo dice su nombre, este campo fue pensado para imponer un limite de tiempo a la vida del paquete en la red. Si no llegaba el paquete antes de ese tiempo, el mismo era descartado por el correspondiente hop. Sin embargo, en la practica esto se implemento como un limite a la cantidad de hops que un paquete podía recorrer. Esto se ve en los headers de los paquetes IPv6, donde el campo fue renombrado como Hop Limit.

En los paquetes IP/ICMP, al header de IP se le suma el header de ICMP. El mismo tiene la siguiente estructura:

\begin{figure}[H]
  \vspace{2em}
  \begin{center}
    \begin{bytefield}[bitwidth=1.1em]{32}
      \bitheader{0-31} \\
      \bitbox{8}{Type} & \bitbox{8}{Code} & \bitbox{16}{Header Checksum} \\
      \bitbox{16}{Identifier} & \bitbox{16}{Sequence Number} \\
      \bitbox{32}{\emph{Datos}}
    \end{bytefield}
  \end{center}
  \caption{Paquete ICMP}
  \label{fig:icmp-header}
\end{figure}

El protocolo ICMP es parte del Internet Protocol Suite, especificado en RFC 792. Este header de 12 bytes se ubica luego del header de IP, teniendo un tamaño de header total de 36 bytes. La especificación explica en detalle como se utiliza el protocolo normalmente. Lo bueno de este protocolo es que si en algún momento header de IP llega a un $ttl = 0$, el hop correspondiente devolverá un mensaje de error al cliente de origen. En unos momentos veremos porque esto es sumamente útil.

En nuestro caso, los siguientes casos de type seran los mas relevantes:

\begin{itemize}
	\item 0 – Echo Reply
	\item 3 - Destination Unreachable
	\item 8 – Echo Request
	\item 11 – Time Exceeded
\end{itemize}

En principio, implementaremos nuestra propia herramienta traceroute utilizando paquetes ICMP. Para ello, enviaremos un Echo Request al URL (Uniform Resource Locator) al que queremos acceder y encontrar la traza utilizando el campo TTL del header IP.

Si el paquete llega al destino, el servidor nos enviara un paquete ICMP con el type Echo Reply. Para poder encontrar los diferentes hops en la traza, utilizaremos el parámetro Time To Live del header IP, inicializándolo en 1 y luego aumentándolo hasta que nos respondan con un Echo Reply. Es decir, si la red hasta el destino tiene 10 hops y queremos identificar el hop $i$, tendremos que settear el parametro $TTL = i$.

Cuando un paquete llega a destino, el servidor nos envia otro paquete con. Sin embargo, cuando falla blah.

%Sacar formato de aca: https://en.wikipedia.org/wiki/Internet_Control_Message_Protocol#Control_messages

Notar que no necesariamente el servidor estara disponible o aceptara paquetes ICMP, por lo que tendremos que poner un limite a la cantidad de hops que buscaremos. Caso contrario iteraríamos hasta llegar al limite dado por los 8 bytes del campo TTL, lo cual no tiene sentido practico. Este procedimiento se puede ver un poco mejor en la siguiente imagen:

\includegraphics[width=\textwidth,keepaspectratio]{images/traceroute_typical}

Para implementar esta herramienta utilizaremos Python, con la libreria scapy. Esta librería nos permite formar paquetes ICMP y luego hacer los respectivos requests.

Notar que el request lo mandaremos a un URL, por lo que el mismo se debe resolver a un IP mediante un request a un DNS (Domain Name System). Correr dos veces la herramienta no nos garantizara que hagamos el request a un mismo IP, dado que un sitio puede tener varios IPs asignados. Esto pasa normalmente con google.com. A su vez, en el ejemplo ilustrativo consideramos una topologia de red sumamente simple. Dado que las topologias tienden a ser sumamente complejas, esto lleva a que al hacer el traceroute se puedan presentar una serie de problemas que deben ser tenidos en cuenta.

\subsection{Anomalidades en traceroutes}

A continuación veremos las potenciales problemáticas de hacer un traceroute utilizando paquetes ICMP. En general las mismas surgen debido a la complejidad innata de las topologías de red. Las mismas en general se pueden agrupar en los siguientes tipos:

\begin{enumerate}
	\item Missing hops
	\item Missing destination
	\item False RRTs
	\item Missing links
	\item Loops and Circles
	\item Diamonds
\end{enumerate}

\newpage
\input{desarrollo}
\newpage
\section{Traceroute a Universidades}

\subsection{google.com}

\begin{table}[H]
\centering
\begin{tabular}{@{}lllll@{}}
\toprule
Hop & Avg. RTT & IP Address & Host name & Location\\ \midrule
1 & 10.6688 & 181.169.12.1 ms & 1-12-169-181.fibertel.com.ar & AR, SA\\
2 &  * * * * * &  &  &  \\
3 &  * * * * * &  &  &  \\
4 &  * * * * * &  &  &  \\
5 & 20.2096 & 200.89.160.21 ms & 21-160-89-200.fibertel.com.ar & AR, SA\\
6 & 14.3278 & 200.89.165.129 ms & 129-165-89-200.fibertel.com.ar & AR, SA\\
7 & 12.5566 & 200.89.165.150 ms & 150-165-89-200.fibertel.com.ar & AR, SA\\
8 &  * * * * * &  &  &  \\
9 & 10.9052 & 209.85.251.86 ms & 209.85.251.86 & US, NA\\
10 & 40.759 & 209.85.252.42 ms & 209.85.252.42 & US, NA\\
11 & 38.5816 & 216.239.58.221 ms & 216.239.58.221 & US, NA\\
12 & 38.1802 & 216.58.202.4 ms & gru06s26-in-f4.1e100.net & US, NA\\ \bottomrule
\end{tabular}
\caption{traceroute: google.com sin caching}
\label{google}
\end{table}


\begin{table}[H]
\centering
\begin{tabular}{@{}lllll@{}}
\toprule
Hop & Avg. RTT & IP Address & Host name & Location\\ \midrule
1 & 11.1854 & 181.169.12.1 ms & 1-12-169-181.fibertel.com.ar & AR, SA\\
2 &  * * * * * &  &  &  \\
3 &  * * * * * &  &  &  \\
4 &  * * * * * &  &  &  \\
5 & 21.9184 & 200.89.165.33 ms & 33-165-89-200.fibertel.com.ar & AR, SA\\
6 & 15.066 & 200.89.164.26 ms & 26-164-89-200.fibertel.com.ar & AR, SA\\
7 &  * * * * * &  &  &  \\
8 & 11.6574 & 181.30.241.187 ms & 187-241-30-181.fibertel.com.ar & AR, SA\\ \bottomrule
\end{tabular}
\caption{traceroute: google.com con caching}
\label{googlecache}
\end{table}

\subsection{dc.ubar.ar}

\begin{table}[H]
\centering
\begin{tabular}{@{}lllll@{}}
\toprule
Hop & Avg. RTT & IP Address & Host name & Location\\ \midrule
1 & 9.3842 & 181.169.12.1 ms & 1-12-169-181.fibertel.com.ar & AR, SA\\
2 &  * * * * * &  &  &  \\
3 &  * * * * * &  &  &  \\
4 &  * * * * * &  &  &  \\
5 & 14.025 & 200.89.164.53 ms & 53-164-89-200.fibertel.com.ar & AR, SA\\
6 & 14.7514 & 200.89.165.2 ms & 2-165-89-200.fibertel.com.ar & AR, SA\\
7 & 22.5916 & 200.89.165.86 ms & 86-165-89-200.fibertel.com.ar & AR, SA\\
8 & 16.5408 & 200.49.69.161 ms & VPN-corp.metrored.net.ar & AR, SA\\
9 &  * * * * * &  &  &  \\
10 &  * * * * * &  &  &  \\
11 &  * * * * * &  &  &  \\
12 & 12.7052 & 157.92.47.53 ms & 157.92.47.53 & AR, SA\\
13 & 13.067 & 192.168.121.2 ms & 192.168.121.2 &  \\
14 &  * * * * * &  &  &  \\
15 &  * * * * * &  &  &  \\

 \bottomrule
\end{tabular}
\caption{traceroute: dc.uba.ar}
\label{dc}
\end{table}

\subsection{mit.edu}

\begin{table}[H]
\centering
\begin{tabular}{@{}lllll@{}}
\toprule
Hop & Avg. RTT & IP Address & Host name & Location\\ \midrule
1 & 12.6968 & 181.169.12.1 ms & 1-12-169-181.fibertel.com.ar & AR, SA\\
2 &  * * * * * &  &  &  \\
3 &  * * * * * &  &  &  \\
4 &  * * * * * &  &  &  \\
5 & 20.0602 & 200.89.160.9 ms & 9-160-89-200.fibertel.com.ar & AR, SA\\
6 & 18.026 & 200.89.165.198 ms & 198-165-89-200.fibertel.com.ar & AR, SA\\
7 & 13.8548 & 200.89.165.86 ms & 86-165-89-200.fibertel.com.ar & AR, SA\\
8 & 13.0754 & 195.22.220.154 ms & xe-1-2-0.baires3.bai.seabone.net & IT, EU\\
9 & 251.8128 & 149.3.183.73 ms & 149.3.183.73 & IT, EU\\
10 & 254.8316 & 89.221.43.107 ms & akamai-row.londra32.lon.seabone.net & IT, EU\\
11 & 253.6456 & 104.65.21.108 ms & a104-65-21-108.deploy.static.akamaitechnologies.com & NL, EU\\\bottomrule
\end{tabular}
\caption{traceroute: mit.edu}
\label{mit}
\end{table}

\subsection{ox.ac.uk}

\begin{table}[H]
\centering
\begin{tabular}{@{}lllll@{}}
\toprule
Hop & Avg. RTT & IP Address & Host name & Location\\ \midrule
1 & 10.9412 & 181.169.12.1 ms & 1-12-169-181.fibertel.com.ar & AR, SA\\
2 &  * * * * * &  &  &  \\
3 &  * * * * * &  &  &  \\
4 &  * * * * * &  &  &  \\
5 & 16.9558 & 200.89.160.13 ms & 13-160-89-200.fibertel.com.ar & AR, SA\\
6 & 15.4314 & 200.89.165.250 ms & 250-165-89-200.fibertel.com.ar & AR, SA\\
7 & 9.7228 & 190.216.88.33 ms & 190.216.88.33 & AR, SA\\
8 & 138.7252 & 67.17.99.233 ms & ae0-300G.ar5.MIA1.gblx.net & US, NA\\
9 &  * * * * * &  &  &  \\
10 &  * * * &  &  &  \\
10 & 224.1195 & 4.69.143.190 ms & ae-1-3104.ear2.London2.Level3.net & US, NA\\
11 & 224.8286 & 212.187.139.166 ms & unknown.Level3.net & GB, EU\\
12 & 236.9458 & 146.97.33.2 ms & ae29.londpg-sbr2.ja.net & GB, EU\\
13 & 240.9694 & 146.97.37.194 ms & ae19.readdy-rbr1.ja.net & GB, EU\\
14 & 227.1278 & 193.63.108.94 ms & ae2.oxfoii-rbr1.ja.net & GB, EU\\
15 & 227.3266 & 193.63.108.98 ms & ae3.oxforq-rbr1.ja.net & GB, EU\\
16 & 228.0936 & 193.63.109.90 ms & 193.63.109.90 & GB, EU\\
17 &  * * * * * &  &  &  \\
18 &  * * * * * &  &  &  \\
19 & 239.6874 & 192.76.32.62 ms & boucs-lompi1.sdc.ox.ac.uk & GB, EU\\
20 & 225.6974 & 129.67.242.154 ms & aurochs-web-154.nsms.ox.ac.uk & GB, EU\\ \bottomrule
\end{tabular}
\caption{traceroute: ox.ac.uk (oxford)}
\label{oxford}
\end{table}

\subsection{u-tokyo.ac.jp}

\begin{table}[H]
\centering
\begin{tabular}{@{}lllll@{}}
\toprule
Hop & Avg. RTT & IP Address & Host name & Location\\ \midrule
1 & 9.9508 & 181.169.12.1 ms & 1-12-169-181.fibertel.com.ar & AR, SA\\
2 &  * * * * * &  &  &  \\
3 &  * * * * * &  &  &  \\
4 &  * * * * * &  &  &  \\
5 & 16.979 & 200.89.160.21 ms & 21-160-89-200.fibertel.com.ar & AR, SA\\
6 & 15.2796 & 200.89.165.222 ms & 222-165-89-200.fibertel.com.ar & AR, SA\\
7 & 10.541 & 195.22.220.102 ms & xe-1-0-3.baires5.bai.seabone.net & IT, EU\\
8 & 39.8348 & 195.22.219.17 ms & ae7.sanpaolo8.spa.seabone.net & IT, EU\\
9 & 36.1798 & 195.22.219.17 ms & ae7.sanpaolo8.spa.seabone.net & IT, EU\\
10 & 42.7854 & 149.3.181.65 ms & 149.3.181.65 & IT, EU\\
11 & 159.2136 & 129.250.2.227 ms & ae-4.r24.nycmny01.us.bb.gin.ntt.net & US, NA\\
12 & 237.3446 & 129.250.4.13 ms & ae-2.r20.sttlwa01.us.bb.gin.ntt.net & US, NA\\
13 & 225.4494 & 129.250.2.54 ms & ae-0.r21.sttlwa01.us.bb.gin.ntt.net & US, NA\\
14 & 426.808 & 129.250.3.86 ms & ae-2.r20.osakjp02.jp.bb.gin.ntt.net & US, NA\\
15 & 429.0596 & 129.250.6.188 ms & ae-4.r22.osakjp02.jp.bb.gin.ntt.net & US, NA\\
16 & 421.2708 & 129.250.2.255 ms & ae-1.r01.osakjp02.jp.bb.gin.ntt.net & US, NA\\
17 & 417.919 & 61.200.80.218 ms & xe-0-4-0-7.r01.osakjp02.jp.ce.gin.ntt.net & JP, AS\\
18 & 425.9262 & 158.205.192.173 ms & ae0.ostcr01.idc.jp & JP, AS\\
19 & 426.6464 & 158.205.192.86 ms & 158.205.192.86 & JP, AS\\
20 & 534.723 & 158.205.121.250 ms & po2.l321.fk1.eg.idc.jp & JP, AS\\
21 & 436.512 & 154.34.240.254 ms & 154.34.240.254 & JP, AS\\
22 & 424.7352 & 210.152.135.178 ms & 210.152.135.178 & JP, AS\\
 \bottomrule
\end{tabular}
\caption{traceroute: u-tokyo.ac.jp}
\label{tokyo}
\end{table}


\begin{enumerate}
	\item Discutir que el DC no hace replies a ICMP.
	\item Discutir enlances transatlanticos
	\item Buscar los hosts y contar un poco que son
	\item Discutir average RTT (5 muestras)
	\item Discutir caching para google. No se conecta directo. Parece que se hace via DNS? Por ahi no, mirar. Siempre se trata de conectar a un IP diferente aparte.
	\item Complementar con un visual traceroute de algun tipo?
\end{enumerate}
\newpage
\section{Experimentos}

\subsection{Caching}

\begin{table}[H]
\centering
\begin{tabular}{@{}lllll@{}}
\toprule
Hop & Avg. RTT & IP Address & Host name & Location\\ \midrule
1 & 10.6688 & 181.169.12.1 ms & 1-12-169-181.fibertel.com.ar & AR, SA\\
2 &  * * * * * &  &  &  \\
3 &  * * * * * &  &  &  \\
4 &  * * * * * &  &  &  \\
5 & 20.2096 & 200.89.160.21 ms & 21-160-89-200.fibertel.com.ar & AR, SA\\
6 & 14.3278 & 200.89.165.129 ms & 129-165-89-200.fibertel.com.ar & AR, SA\\
7 & 12.5566 & 200.89.165.150 ms & 150-165-89-200.fibertel.com.ar & AR, SA\\
8 &  * * * * * &  &  &  \\
9 & 10.9052 & 209.85.251.86 ms & 209.85.251.86 & US, NA\\
10 & 40.759 & 209.85.252.42 ms & 209.85.252.42 & US, NA\\
11 & 38.5816 & 216.239.58.221 ms & 216.239.58.221 & US, NA\\
12 & 38.1802 & 216.58.202.4 ms & gru06s26-in-f4.1e100.net & US, NA\\ \bottomrule
\end{tabular}
\caption{traceroute: google.com sin caching}
\label{google}
\end{table}


\begin{table}[H]
\centering
\begin{tabular}{@{}lllll@{}}
\toprule
Hop & Avg. RTT & IP Address & Host name & Location\\ \midrule
1 & 11.1854 & 181.169.12.1 ms & 1-12-169-181.fibertel.com.ar & AR, SA\\
2 &  * * * * * &  &  &  \\
3 &  * * * * * &  &  &  \\
4 &  * * * * * &  &  &  \\
5 & 21.9184 & 200.89.165.33 ms & 33-165-89-200.fibertel.com.ar & AR, SA\\
6 & 15.066 & 200.89.164.26 ms & 26-164-89-200.fibertel.com.ar & AR, SA\\
7 &  * * * * * &  &  &  \\
8 & 11.6574 & 181.30.241.187 ms & 187-241-30-181.fibertel.com.ar & AR, SA\\ \bottomrule
\end{tabular}
\caption{traceroute: google.com con caching}
\label{googlecache}
\end{table}

\subsection{Detección de links intercontinentales}

1. Falsos Positivos / Falsos Negativos


            Intercontinental     Local
Test Intercontinental
Test Local

Muestra: 100 sitios de alexa?

Hacer funcion que detecte enlaces intercontinentales con libreria de Python.

\subsection{Traceroute anomalities}


\newpage
\section{Conclusión}

Poder trazar la ruta que un paquete potencialmente puede tomar es sumamente importante, particularmente para podes identificar saltos donde la red se puede congestionar o tener errores. Esta funcionalidad es ampliamente utilizada a día de hoy, si bien en este trabajo practico vimos un método para la implementación de esta herramienta, existen otras maneras de implementarla.

El método propuesto en el TP consiste en abusarse de los paquetes ICMP para poder obtener la ruta del origen al destino, el problema de esto es que es posible que la ruta cambie durante la ejecución obteniendo así una ruta invalida (es una posibilidad minima que puede ser mitigada fácilmente ejecutando varias veces el algoritmo), en varias ocasiones se contemplo incluir la traza de rutas en los protocolos y paquetes, un método fue propuesto en el RFC 791 (\textit{record route}) y el RFC 1393 (\textit{traceroute con IP Option}). El primer método consiste en guardar las direcciones de 32 bits en el header del paquete a medida que transita la red, mientras que el segundo consiste en que los enrutadores sepan que se esta haciendo un traceroute y envíen paquetes identificándose como saltos al origen, el método con \textit{record route} existe aun a día de hoy pero esta limitado a guardar a lo sumo 9 saltos por limitaciones de espacio en el header (creemos que esto se debe a que fue planteado en 1981, antes de la Internet moderna), mientras que el \textit{traceroute con IP Option} requería que todos los enrutadores soporten la opción IP apropiada, así que como era de esperar nunca fue implementado y fue deprecado en 2012 por el RFC 6814.

%Desde un marco "`legal"', tenemos la opcion de \textit{record route} y 
%Discutir alternativas, onda hacer esto por IP (IPv6 tiene la opcion de hacer un traceroute, esta en los slides de clase del tp!). Hablar de como cambia la cantidad de paquetes que deben ser enviados. Fijarse si existen otros metodos!

Respecto a los resultados obtenidos con nuestra herramienta, casi todos se correspondieron con lo que esperábamos. Sin embargo, el de Japón fue el mas interesante, mirando el mapa de enlaces intercontinentales podemos ver que la manera mas directa hacia el destino es a través del océano pacifico, empleando los enlaces que hay sobre la costa oeste de Estados Unidos. Para poder llegar a dichos enlaces hay varias formas, si bien se tomo un camino por Europa, también es posible tomar varios caminos desde Argentina hacia la costa este de Estados Unidos, y luego recorrer hacia la costa oeste de dicho país. Lo mas importante a destacar es que esto deja en claro que la distribución de fibra óptica a lo largo del mundo esta pensada en base al trafico mas que en minimizar la latencia a nivel mundial, pero al igual que dijimos durante el análisis, esto tiene sentido considerando lo reducido que debe ser el trafico de Argentina hacia Japón.

%Comentar sobre la ruta a Japon, que es la que mas sorprende.

De los dos métodos discutidos, ambos tuvieron sus falencias. Esta claro que el método infalible seria un base de datos como la de GeoIP que este constantemente actualizada, eso permitiría erradicar toda ambigüedad, sin embargo, la naturaleza dinámica y la diversidad burocrática de la Internet hace que sea imposible tener dicha información actualizada, lo cual llevo a los errores vistos. El método de Simbala, por otro lado, es susceptible a la congestión de la red ya que se basa en outliers y en mediciones de tiempo, esto también lo hace sumamente débil ya que si el servidos no responde a los paquetes de \textit{Time Exceeded} el método no se puede aplicar sobre ese salto, sin embargo, cuando la información estaba disponible los resultados fueron aceptables. En la practica, lo prudente seria aplicar varios criterios distintos para evaluar si en un salto hubo un enlace intercontinental, y luego tomar una decisión según la fiabilidad de cada unos de los criterios en cuestión.


 
%Discutir pros and cons de Simbala vs GeoIP. Como se podria mejorar y complementar.

%Charlar sobre el uso de embebidos para network topology (discutir challenges de topology). Meter un par de imagenes copadas del paper? cerrar con ideas, estadisticas e imagenes de aca? http://internetcensus2012.bitbucket.org/paper.html
\newpage
\bibliographystyle{plain}
\nocite{*}
\bibliography{bibliografia}

\end{document}